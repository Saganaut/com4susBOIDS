\documentclass[11pt]{article}

\usepackage{amssymb}
\usepackage{amsmath}
\usepackage{graphicx}
\usepackage{cite}
\usepackage{todonotes}
\usepackage{url}
\usepackage{tikz}
\usetikzlibrary{arrows}

\usepackage{multicol}
\usepackage{caption}
\usepackage{subcaption}

\setlength{\paperwidth}{8.5in}
\setlength{\paperheight}{11in}
\setlength{\voffset}{-0.2in}
\setlength{\topmargin}{0in}
\setlength{\headheight}{0in}
\setlength{\headsep}{0in}
\setlength{\footskip}{30pt}
\setlength{\textheight}{9.25in}
\setlength{\hoffset}{0in}
\setlength{\oddsidemargin}{0in}
\setlength{\textwidth}{6.5in}
\setlength{\parindent}{0in}
\setlength{\parskip}{9pt}

\newcommand{\tod}[1]{\textcolor{red}{[#1]}}
\newcommand{\ben}{\begin{enumerate}}
\newcommand{\een}{\end{enumerate}}

\DeclareGraphicsRule{.JPG}{eps}{*}{`jpeg2ps #1}

\title{Mining eBird Networks\\ Project Proposal\vspace{-8pt}}
\author{Casey Battaglino\\Robert Pienta}
\date{}
\begin{document}
\maketitle
\begin{multicols}{2}
\section{Introduction} \vspace{-10 pt}
%\tod{Birds are beautiful and we need to understand more about how they interact as seasonally transient members of our ecosystem.}
eBird\cite{DBLP:conf/iaai/KellingGFLWYDG12} is a `citizen science' project that allows for users to report and store bird sightings online. This allows for the crowd-sourcing of bird identification and monitoring, a process that would otherwise require massive amounts of time and effort from researchers. 

While this removes the ``birden'' of data collection from experts, it also introduces a number of new problems. In particular, it introduces a higher probability of misidentification (when non-experts attempt to identify birds), and naturally produces more data in areas of higher human population density (the problem of spatial bias). Thus, data must be sanitized and normalized before it can be used to actually model bird presence. Solutions to these particular problems are presented by Kelling, et. al \cite{DBLP:conf/iaai/KellingGFLWYDG12}.

The problem of generating a more continuous (spatially and temporal) species distribution from this data is discussed in the `STEM' paper of Fink, et. al~\cite{stem}.

Thus, a good deal of research has been put into generating accurate species distribution estimates over time. We are interested in using this generated data to explore higher-level questions regarding the movement of migratory birds. 

In particular, we are interested in inferring an underlying migration network of a given species of bird, given its spatial distribution over time. In our proposed model, nodes in the network represent a contiguous area of the United States, and weighted edges in the network represent the computed confidence of bird migration from location to location, weighted by the confidence that birds are migrating from one location to the other. This network could then be used to automatically identify important areas (high-flow bottlenecks) for bird migration, as well as important sources and sinks of migration. 

\section{Proposed Approach} \vspace{-10 pt}
\subsection{Data}
Our proposed approach relies on the availability of clean, reliable data for bird distribution from eBird. Through correspondence with Prof. Dilikina, we will have access to spatio-temporal data for two species of migratory birds, generated by the STEM model~\cite{stem}. 

This data comes in the form of spatio-temporal data at hundreds of thousands of points distributed through the United States. We propose to reduce noise and computational complexity by aggregating these points into a smaller number of coarse points that aggregate the time-series data for all fine-grained points within a certain radius. This process may require proposing an interpolation function. The radius that a coarse point covers then becomes a tuning parameter in our final model. (A large radius will reduce noise at the expense of a less-detailed geographical model). 

After analyzing the data, we will have a set of nodes that are spatially distributed through the United States, which have time-series data (concentration of a particular bird species at time $t$) associated with each node. As stated, our goal is network inference --- we wish to find a network that represents bird migration through these points in a given time slice. In the naive case, we would have to compare the time series data between every two points, for a total of $O(n^2)$ time-series comparisons

Fortunately, we only care about constructing a network that joins nodes that are geographically close (because it is infeasible for birds to instantly travel large distances). Thus, we construct an overlay network out of the nodes where any two nodes within a certain geographical distance are connected by an edge --- the presence of an edge in this overlay network means that we will consider it as a candidate for network inference, and perform a computation between the two nodes. Because this overlay network is embedded on a 2D plane, this means we only need to make $O(n)$ time-series comparisons (provided our `closeness' criteria is a small function of our `coarseness' criteria). 

\subsection{Network Inference} 
Statistical approaches to graph inference is a well researched area with many effective techniques \cite{AlbertMechanics}. Because the nature of this research problems involves inferring a network from a time-varying signal, we will leverage the work of Kramer et al. who proposed a state of the art method for statistically inferring an underlying functional network from a time-varied signal \cite{kramer}. In this case, we attempt to establish the existence of signal propagation from one node to another, with a certain degree of confidence. 

Kramer, et. al present three methods of increasing sophistication (and increasing computational intensity) for inferring a network from a set of nodes and corresponding time-series data. While the third method looks like it may be computationally infeasible (requiring a massive number of FFTs compounded by a massive number of bootstrap runs), we will explore the effectiveness of all three methods in our work. 

A final detail is that we will need to look at `slices' of the time-series data and perform network inference for a set number of time periods (to account for different migratory behaviors in different seasons). The final network can be unweighted (where an edge exists if its weight is above a certain threshold), or weighted by the confidence that it is in the migratory network, to identify important corridors, sources, and sinks using a flow analysis. 

\section{Related Work}
The data from eBird was already used by Hurlbert et al. in an investigation of red vireo migration; wherein they discovered a direct correlation between the migration patterns and the average temperature \cite{hurlbert}. 

Another migration study, using Hidden Markov models on synthetic data, was performed by Sheldon, et. al \cite{conf/nips/SheldonEK07}. We plan to compare and contrast our approach to these studies, and any others that we find.

\section{Proposed Timeline} \vspace{-10 pt}
The following table denotes our estimates for the \textit{time of completion} for various stages of this research.
\begin{table*}
\centering
\begin{tabular}{|ll|} \hline
Task & Date\\ \hline
Deep Literature Review & March 12  \\
Procure eBird Data & March 14 \\
eBird Data Cleaning, Denoising, and Normalization. & March 15 \\
Implementation: Network Edge Inferer (NEI) &March 24 \\
Testing of NEI & March 28  \\
Application of NEI on eBird Data & April 1 \\

\hline\end{tabular}
\caption{The estimation of task completion dates for our eBird research.}
\label{table:chart}
\end{table*}

\end{multicols}
\bibliographystyle{plain}
\bibliography{bib}


\end{document}