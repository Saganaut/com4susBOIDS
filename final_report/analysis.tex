\section{Conclusion}
As is clear from the results, we were able to generate concrete results and map them in a geographical context. One outstanding question is how highly-correlated our model is with the actual bird-presence data. Cross-correlation on its own will find two flat signals to be highly correlated, so it is clearly not a sufficient metric on its own. Our work-around to this problem was by adding a `movement' model that we then filter by. By computing an average-absolute derivative, we were able to remove from consideration nodes where bird population did not change much within a given time window. 

From the results, we still qualitatively see a high correlation between bird presence and predicted migration networks. Also troublesome are the vast swaths of the northern US that are predicted as a migration network during the summer months. Much of this could be minimized with further tuning (filtering out even more nodes based on derivative). 

One encouraging feature of our model's output is that it ignores the population of tree swallows in California for a good part of the year. This corresponds with the general knowledge that the tree swallow is a breeding resident of California, not a passage-migrant. 

\section{Future Work}
Qualitatively analyzing networks gives a limited picture. Quantitative measures can leverage the data more deeply. While betweenness centrality could identify some important `bottleneck' nodes, we believe that a maximum-flow measure would be more suitable, as this could also identify important sources and sinks if modeled correctly.

The most important future work, were this to become a publication, is validation of the produced data with well-known existing migration patterns, as well as further analysis with a wide spread of bird species. Generated networks are still strongly correlated with the actual bird presence data, so it's quite possible that there are far better metrics than cross-correlation to establish edge weights. Important insights could most likely be drawn from the signal-processing community. 