\documentclass[11pt]{article}

\usepackage{amssymb}
\usepackage{amsmath}
\usepackage{graphicx}
\usepackage{cite}
%\usepackage{algorithmic}
%\usepackage{algorithm}
\usepackage{todonotes}
\usepackage{url}
\usepackage{tikz}
\usetikzlibrary{arrows}

\usepackage{multicol}
\usepackage{caption}
\usepackage{subcaption}


\usepackage{listings}
 \lstset{
            language=Matlab,                                % choose the language of the code
    %       basicstyle=10pt,                                % the size of the fonts that are used for the code
            numbers=left,                                   % where to put the line-numbers
            numberstyle=\footnotesize,                      % the size of the fonts that are used for the line-numbers
            stepnumber=1,                                           % the step between two line-numbers. If it's 1 each line will be numbered
            numbersep=5pt,                                  % how far the line-numbers are from the code
    %       backgroundcolor=\color{white},          % choose the background color. You must add \usepackage{color}
            showspaces=false,                               % show spaces adding particular underscores
            showstringspaces=false,                         % underline spaces within strings
            showtabs=false,                                         % show tabs within strings adding particular underscores
    %       frame=single,                                           % adds a frame around the code
    %       tabsize=2,                                              % sets default tabsize to 2 spaces
    %       captionpos=b,                                           % sets the caption-position to bottom
            breaklines=true,                                        % sets automatic line breaking
            breakatwhitespace=false,                        % sets if automatic breaks should only happen at whitespace
            escapeinside={\%*}{*)}                          % if you want to add a comment within your code
}

\setlength{\paperwidth}{8.5in}
\setlength{\paperheight}{11in}
\setlength{\voffset}{-0.2in}
\setlength{\topmargin}{0in}
\setlength{\headheight}{0in}
\setlength{\headsep}{0in}
\setlength{\footskip}{30pt}
\setlength{\textheight}{9.25in}
\setlength{\hoffset}{0in}
\setlength{\oddsidemargin}{0in}
\setlength{\textwidth}{6.5in}
\setlength{\parindent}{0in}
\setlength{\parskip}{9pt}

\newcommand{\tod}[1]{\textcolor{red}{[#1]}}
\newcommand{\ben}{\begin{enumerate}}
\newcommand{\een}{\end{enumerate}}

\DeclareGraphicsRule{.JPG}{eps}{*}{`jpeg2ps #1}

\title{Mining eBird Networks\\ Project Proposal\vspace{-8pt}}
\author{Casey Battaglino\\Robert Pienta}
\date{}
\begin{document}
\maketitle
\begin{multicols}{2}
\section*{Introduction} \vspace{-10 pt}

\tod{Birds are beautiful and we need to understand more about how they interact as seasonally transient members of our ecosystem.}


We are interested in inferring the underlying network of a single species of bird distributed both over time and geographically across the United States.

\tod{Needs for careful normalization and data-cleaning if raw eBird data is used.}

\section*{Proposed Approach} \vspace{-10 pt}
\tod{Data preprocessing, cleaning, de-noising}


\tod{Time window based network construction.}


\tod{Spatial network evolution and bird migration}
The data from eBird was already used by Hurlbert et al. in an investigation of red vireo migration; wherein they discovered a direct correlation between the migration patterns and the average temperature \cite{hurlbert}. 

\section*{Relevant Literature}
Statistical approaches to graph inference is a well researched area with many effective techniques \cite{AlbertMechanics}.
Because the nature of this research problems involves inferring a network from a time-varying signal, we will leverage the work of Kramer et al. 
Kramer et al. proposed a state of the art method for statistically inferring an underling network from a time-varied signal \cite{kramer}.

\section*{Proposed Timeline} \vspace{-10 pt}
The following table denotes our estimates for the \textit{time of completion} for various stages of this research.
\begin{table*}
\centering
\begin{tabular}{|ll|} \hline
Task & Date\\ \hline
Deep Literature Review & March 12  \\
Procure eBird Data & March 14 \\
eBird Data Cleaning, Denoising, and Normalization. & March 15 \\
Implementation: Network Edge Inferer (NEI) &March 24 \\
Testing of NEI & March 28  \\
Application of NEI on eBird Data & April 1 \\

\hline\end{tabular}
\caption{The estimation of task completion dates for our eBird research.}
\label{table:chart}
\end{table*}






\end{multicols}
\bibliographystyle{plain}
\bibliography{bib}


\end{document}